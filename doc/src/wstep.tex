\section{Wstęp}
\indent

Celem projektu przedstawionego w~poniższym sprawozdaniu było stworzenie modelu
układu chłodzenia cieczą oraz oprogramowania do nadzorowania tego procesu
--~systemu SCADA. W~ramach zrealizowanych dotychczas zajęć możliwe było
zapoznanie się z~profesjonalnymi narzędziami służącymi do kontrolowania
przebiegu procesu na przykładzie oprogramowania \textit{zenon}.

Posiadając już doświadczenie w~obsłudze dedykowanych rozwiązań, które mimo
swojego zaawansowania sprawiają niekiedy wrażenie, czasami aż nadmiernie,
skomplikowanych i~topornych, nadrzędnym zadaniem stała się próba wykonania
panelu operatorskiego z~wykorzystaniem narzędzi typowo programistycznych, czyli
języka \textit{C++} oraz biblioteki graficznej \textit{Qt}. Możliwość
samodzielnego wpływania na budowę i~złożoność tworzonego oprogramowania oraz
wszystkie jego funkcjonalności była bardzo zachęcającym argumentem przy
podejmowaniu wyboru odnośnie tematu poniższej pracy. W~związku z~tym, w~wielu
miejscach będą pojawiały się porównania do pakietu \textit{zenon}, głównie
dotyczące aspektów obsługi.

Dodatkowym elementem prac było wykorzystanie płytki ewaluacyjnej firmy
\textit{STMicroelectronics} do symulowania sterownika modelowanego układu.
Również w~tym miejscu pojawił się pomysł przetestowania alternatywnych,
relatywnie prostszych w~porównaniu do przemysłowych, protokołów komunikacyjnych
pomiędzy komputerem a sterownikiem.

Najważniejszym czynnikiem decydującym o~powodzeniu projektu była poprawność
działania oraz stabilność stworzonego oprogramowania, co również zostało
przetestowane i~opisane w~odpowiednich rozdziałach poniższego sprawozdania.
