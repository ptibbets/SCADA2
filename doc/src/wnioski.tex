\section{Wnioski}
\indent

Cel opisanego we wcześniejszych rozdziałach projektu, którym było stworzenie
symulatora układu chłodzenia cieczą procesora komputera oraz stworzenie panelu
operatorskiego z~wykorzystaniem biblioteki \textit{Qt} w~języku \textit{C++}
może być z~pewnością uznany za osiągnięty.

Praca nad tym projektem pozwoliła spojrzeć w~zupełnie inny sposób na tworzenie
oprogramowania typu SCADA, gdyż to na autorach spoczywała odpowiedzialność za
wprowadzenie odpowiedniej funkcjonalności do programu oraz zapewnienie
skuteczności jego działania. W~trakcie prac udało się zauważyć zarówno zalety,
jak i~pewne wady oprogramowania specjalistycznego. Do tych pierwszych zaliczały
się na pewno:
\begin{itemize}
    \item możliwość utworzenia panelu operatorskiego dla dowolnego procesu,
    \item kompatybilność z~dużą ilością sprzętu dostępnego na rynku,
    \item łatwość zmiany konfiguracji utworzonego programu.
\end{itemize}
Jednak część z~tych zalet może bardzo łatwo obrócić się w~wady, przynajmniej dla
użytkownika, który szuka prostszych rozwiązań na potrzeby kontrolowania
niewielkiego i~nieskomplikowanego procesu technologicznego. Podobne odczucia
posiadali autorzy, gdy odbywali pierwsze zajęcia z~wykorzystaniem programu
\textit{zenon}, gdy mnogość dostępnych opcji utrudniała wielokrotnie
odnalezienie tych właściwych, w~danym momencie wymaganych.

O~ile realizowany projekt był pewnego rodzaju eksperymentem mającym sprawdzić,
jakie nakłady pracy są niezbędne do stworzenia panelu operatorskiego dla
zadanego procesu technologicznego oraz ile problemów może to sprawić, o~tyle
należy w~tym miejscu przyznać, że dzięki znajomości narzędzi, które były
wykorzystywane w~trakcie pracy udało się osiągnąć efekt bez większych problemów
po drodze.

Podsumowując, oprogramowanie typu SCADA stworzone przez wyspecjalizowane firmy
jest potężnym narzędziem pozwalającym na przygotowanie paneli operatorskich oraz
innych narzędzi niezbędnych do prawidłowego zarządzania procesami
technologicznymi. Mimo wszystko, autorzy uważają, że w~wielu przypadkach,
zarówno hobbystycznych, jak i~w~odniesieniu do prostych procesów realizowanych
w~mniejszych lub większych zakładach pracy, dowiedzenie możliwości utworzenia
oprogramowania przygotowanego do pracy w~specyficznych i~z~góry znanych
warunkach jest niewątpliwie bardzo dobrym rezultatem wykonanych prac.
