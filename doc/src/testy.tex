\section{Testy oraz porównania}
\indent

Poniższy rozdział zawiera kilka testów oprogramowania, które zostały
przeprowadzone po zakończeniu prac związanych z~implementacją wszystkich
funkcjonalności. Nie są to na pewno rozważania wszystkich możliwych przypadków,
lecz w~głównej mierze miały one za zadanie sprawdzić działanie podstawowych
elementów stworzonego programu, zarówno po stronie komputera, jak
i~mikrokontrolera.

\subsection*{Działanie symulacji w~czasie rzeczywistym}
\indent

Sprawdzenie, czy symulacja realizowana jest w~czasie rzeczywistym zostało
wykonane przy pomocy narzędzi służących do debugowania kodu. Uruchomiony został
w~ten sposób program na płytce, a~następnie przeprowadzonych zostało około
kilkanaście iteracji głównej pętli programu, która miała za zadanie prowadzić
komunikację z~komputerem oraz rozwiązywać równania różniczkowe.

Dzięki wykorzystaniu wewnętrznego timera układu, który zlicza czas
z~dokładnością do jednej milisekundy udało się ustalić, iż mikrokontroler bez
problemu radzi sobie z~rozwiązywaniem utworzonych równań różniczkowych oraz
komunikacją w~czasie nie przekraczającym $1$~[$ms$], gdyż w~czasie wykonywania
tych operacji nie doszło do przepełnienia licznika timera. Niewątpliwą zasługą
tego stanu rzeczy było relatywnie wysokie taktowanie rdzenia procesora
($80$~[$MHz$]), a~także wykorzystanie sprzętowego koprocesora do wykonywania
operacji zmiennoprzecinkowych.

\subsection*{Niezawodność komunikacji}
\indent

Ten test polegał na sprawdzeniu na ile niezawodna jest komunikacja pomiędzy
urządzeniami z~wykorzystaniem interfejsu szeregowego. Przyrównując ten sposób
komunikacji do łączności sieciowej, przypomina on w~pewnym sensie protokół
\textit{UDP}, przynajmniej w~wykorzystywanej wersji, gdy nie są używane linie
sygnałowe. Porównanie to jest bardzo słuszne, gdyż podobnie jak w~transmisji
z~wykorzystaniem protokołu \textit{UDP} jest ona bezpołączeniowa, a~zatem
wszystkie aspekty odpowiedzialne za poprawność oraz skuteczność transmisji muszą
być wykonywane na wyższej warstwie.

W~badanych sytuacjach transmisja przebiegała prawidłowo, gdy zarówno urządzenie,
jak i~aplikacja komputerowa, zostały uruchomione w~sposób prawidłowy. Nie udało
się w~takim przypadku zaobserwować jakichkolwiek zakłóceń transmisji. Pewne
problemy pojawiły się natomiast w~przypadku chwilowego ,,zawieszenia się''
komputera z~powodu wykonywania jednocześnie innych operacji na nim. Doszło wtedy
najprawdopodobniej do rozsynchronizowania transmisji, przez co aplikacja,
w~obecnej chwili nieprzygotowana w~pełni na taką sytuację, nie poradziła sobie
z~tym i~uległa ,,zawieszeniu''. Niestety nie udało się odtworzyć przedstawionego
przypadku, przez co niemożliwe było zarówno zbadanie jego przyczyn, jak
i~wdrożenie odpowiednich mechanizmów zabezpieczających przed pojawieniem się
podobnych sytuacji w~przyszłości.

\subsection*{Stabilność działania aplikacji komputerowej}
\indent

Pomijając sytuacje, gdy kod programu był w~fazie rozwoju i~testów, nie udało się
zaobserwować nieprzewidywalnego zachowania aplikacji. Zapewne duża w~tym zasługa
bardzo dobrze stworzonej biblioteki \textit{Qt}, która umożliwia relatywnie
łatwe tworzenie programów ze środowiskiem graficznym, a~jednocześnie jest to na
tyle rozbudowane w~ukrytej warstwie tej biblioteki, że odpowiednio podchodząc do
programowania z~jej wykorzystaniem można łatwo unikać popełniania poważniejszych
błędów w~kodzie źródłowym.

\subsection*{Porównanie użytkowania z~oprogramowaniem specjalistycznym}
\indent

Na podstawie doświadczeń wyniesionych z~zajęć, które prowadzone były
z~wykorzystaniem specjalistycznego oprogramowania typu SCADA możliwe było
dokonanie porównania funkcjonalności stworzonej aplikacji z~programem
\textit{zenon}.

Pewne szczególne różnice oraz podobieństwa zostały przedstawione we
wcześniejszych rozdziałach dotyczących specyficznych elementów stworzonego
oprogramowania. W~tym miejscu należy jednak zwrócić szczególną uwagę, że o~ile
program \textit{zenon} oraz jemu podobne służą do tworzenia paneli operatorskich
dla dowolnego procesu technologicznego i~umożliwiają komunikację z~wiekszością
popularnego sprzętu dostępnego na rynku z~wykorzystaniem ustandaryzowanych
protokołów, o~tyle stworzona aplikacja wykonywana była z~myślą o~jednym
konkretnym procesie technologicznym realizowanym na konkretnym urządzeniu, zaś
komunikacja przebiegała z~wykorzystaniem specjalnie przygotowanych na te
potrzeby struktur danych.

Na pewno należy nadmienić, że czas tworzenia oprogramowania pod konkretne
zastosowanie może być znacznie dłuższy, aniżeli tworzenie paneli operatorskich
w~specjalnie do tego celu stworzonych programach, nie mniej jednak nic nie stoi
na przeszkodzie, aby przygotować odpowiednie biblioteki do dowolnego języka
programowania w~celu ułatwienia i~przyspieszenia tego procesu. Zaletą wybranego
podejścia była możliwość dostosowania programu oraz jego zasobów pod konkretne
zastosowanie, przez co użytkownik oraz programista nie miał problemu
z~odnalezieniem się w~dostępnych opcjach konfiguracyjnych oraz funkcjach
bibliotecznych.
